\documentclass[fleqn]{jsarticle}
\usepackage{multicol}
\usepackage{titlesec}

\titleformat*{\section}{\LARGE\bfseries}
\titleformat*{\subsection}{\large\bfseries}

\begin{document}

\begin{flushright}
  Masanari 
\end{flushright}
  \setcounter{section}{1}
\section{二階線形微分方程式}
  \setcounter{subsection}{5}
    \subsection{解の存在と一意性、ロンスキアン}
      このセクションでは線形同時常微分方程式を議論する。 
    \begin{eqnarray}
        y '' + p(x)y' +q(x) = 0 \\[1.5cm]
        y(x_0) = k_0, \hspace{2zw} y'(x_0) = k_1 \\[2cm]
        y = c_1y_1 + c_2y_2 \\[2cm]
        Iでk_1y_1(x)+ k_2y_2(x) =0 はk_1=0, k_2 =0を暗示する \\[2cm] 
        (a) y_1= ky_2, \hspace{2zw} or (b) y_2 =ly_1\\[2cm] 
        W(y_1, y_2)= y_1y'_2 - y_2y'_1 
      \end{eqnarray}
$$        W(y_1,y_2)= y_1y'_2 - y_2y'_1 = ky_2y'_1 - y_2ky'_2 =0 \\[2cm]$$
      \begin{eqnarray}
        k_1y_1(x_0) + k_2y_2(x_0) =0, 
          k_1y'_1(x_0) = k_2y'_2(x_0) = 0
      \end{eqnarray}

$$        k_1y_1(x_0)y'_2(x_0) - k_1y'_1(x_0)y_2)(x_0) = k_1W(y_1(x_0),y_2(x_0)) =0 \\[1.5cm]$$
        


  
\end{document}
